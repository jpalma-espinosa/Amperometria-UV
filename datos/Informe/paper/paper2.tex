\documentclass[11pt,a4paper, twocolum]{article} 
\usepackage[utf8]{inputenc} 

\usepackage[rflt]{floatflt} 


\renewcommand*\familydefault{\sfdefault} 
\usepackage[scaled]{helvet} %para usar helvetica
\renewcommand\familydefault{\sfdefault} 
\usepackage[T1]{fontenc}

\usepackage{graphicx}  
\newcommand{\imgdir}{doc-img} 
\graphicspath{{\imgdir/}} 

\usepackage{amsmath} 
\numberwithin{equation}{section}
\numberwithin{figure}{section} 
\numberwithin{table}{section} 

\usepackage{color} %red, green, blue, yellow, cyan, magenta, black, white
\definecolor{mygreen}{RGB}{28,172,0} % color values Red, Green, Blue
\definecolor{mylilas}{RGB}{170,55,241}



\usepackage{listings}
\lstset{language=Matlab,%
    %basicstyle=\color{red},
    breaklines=true,%
    morekeywords={matlab2tikz},
    keywordstyle=\color{blue},%
    morekeywords=[2]{1}, keywordstyle=[2]{\color{black}},
    identifierstyle=\color{black},%
    stringstyle=\color{mylilas},
    commentstyle=\color{mygreen},%
    showstringspaces=false,%without this there will be a symbol in the places where there is a space
    numbers=left,%
    numberstyle={\tiny \color{black}},% size of the numbers
    numbersep=9pt, % this defines how far the numbers are from the text
    emph=[1]{for,end,break},emphstyle=[1]\color{red}, %some words to emphasise
    %emph=[2]{word1,word2}, emphstyle=[2]{style},    
}

\pagestyle{empty}

%set dimensions of columns, gap between columns, and space between paragraphs
\setlength{\textheight}{8.75in}
\setlength{\columnsep}{2.0pc}
\setlength{\textwidth}{6.8in}
%\setlength{\footheight}{0.0in}
\setlength{\topmargin}{0.25in}
\setlength{\headheight}{0.0in}
\setlength{\headsep}{0.0in}
\setlength{\oddsidemargin}{-.19in}
\setlength{\parindent}{1pc}

%I copied stuff out of art10.sty and modified them to conform to IEEE format

\makeatletter
%as Latex considers descenders in its calculation of interline spacing,
%to get 12 point spacing for normalsize text, must set it to 10 points
\def\@normalsize{\@setsize\normalsize{12pt}\xpt\@xpt
\abovedisplayskip 10pt plus2pt minus5pt\belowdisplayskip \abovedisplayskip
\abovedisplayshortskip \z@ plus3pt\belowdisplayshortskip 6pt plus3pt
minus3pt\let\@listi\@listI} 

%need an 11 pt font size for subsection and abstract headings
\def\subsize{\@setsize\subsize{12pt}\xipt\@xipt}

%make section titles bold and 12 point, 2 blank lines before, 1 after
\def\section{\@startsection {section}{1}{\z@}{24pt plus 2pt minus 2pt}
{12pt plus 2pt minus 2pt}{\large\bf}}

%make subsection titles bold and 11 point, 1 blank line before, 1 after
\def\subsection{\@startsection {subsection}{2}{\z@}{12pt plus 2pt minus 2pt}
{12pt plus 2pt minus 2pt}{\subsize\bf}}
\makeatother

%\graphicspath{{./doc-img/}}
\usepackage{graphicx}

\begin{document}


%don't want date printed
\date{}

%make title bold and 14 pt font (Latex default is non-bold, 16 pt)
\title{\Large\bf Análisis de liberación de catecolaminas en células cromafinas de ratón, mediante técnica de amperometría}

%for single author (just remove % characters)
\author{J. Palma-Espinosa \\
  Electrical Engineering Department \\
  Universidad de Santiago de Chile. Santiago, Chile \\
  javier.palma@usach.cl}
 
%for two authors (this is what is printed)
%\author{\begin{tabular}[t]{c@{\extracolsep{8em}}c}
%  I. M. Author	& M. Y. Coauthor \\
% \\
%  My Department & Coauthor Department \\
%  My Institute & Coauthor Institute \\
%  City, ST~~zipcode	& City, ST~~zipcode
%\end{tabular}}

\maketitle

%I don't know why I have to reset thispagesyle, but otherwise get page numbers
\thispagestyle{empty}

\subsection*{\centering Abstract}
%IEEE allows italicized abstract
{\em
This is the abstract of my paper.  It must fit within the size allowed, which
is about 3 inches, including section title, which is 11 point bold font.  If 
you don't want the text in italics, simply remove the 'em' command and the 
curly braces which bound the abstract text.  If you have em commands within an 
italicized abstract, the text will come out as normal (nonitalicized) text.  
%end italics mode
}

\section{Introduction}
In today’s competitive industrial sector, the manufacturers invest significant amount of money to ensure that the plants function with full capacity and efficiently with zero downtime and without unanticipated output quality. Thus, Selection of the right technology for the required application is challenging for the industrial environment. Furthermore, interferences to the mission-critical data can result in costly disasters in terms of money, manpower, time and even lives of employees or public\cite{low2005wireless}, as it was showed in \cite{bowie}.
For example, the productivity and efficiency of current industrial facilities can be greatly improved by using wireless sensor network technology for remote control and automation as well as for efficient data collection\cite{cheffena2016propagation}.
In wired networks, each sensor node requires a separate twisted shielded pair wire connection. That becomes expensive if the cabling across sensor nodes and the controllers are long and configuration management is difficult. Wireless sensor network (WSN) is relatively very cost efficient. Moreover, its self-configure, self-organizing characteristics make the wireless sensor networks robust and ideal for hazardous plant and high-assets protection applications\cite{low2005wireless}.

As several authors have reported\cite{low2005wireless, cheffena2012industrial, luo2011rf}, the industrial environment is not the best environment for setting up a WSN. The effect of noise due to the broad operating temperatures, heavy machinery, ignition systems, vibration among other usual industrial activities is high.  Also, the interference due to the use of the 2.4GHz Industrial, scientific and medical (ISM) unlicensed frequency by other technologies like Wifi, Bluetooth, wireless USB, microwaves, cordless phones and other sources is an undesired effect that must be also taken into account. If not, the aforementioned propagation effects can significantly degrade the performance of WSNs in industrial settings. Knowledge of the propagation channel is thus required to design and evaluate WSNs for industrial applications\cite{cheffena2016propagation}.

However, the knowledge of the propagation channel is not an easy task.  It will highly depend on the characteristic of the industrial environment where the WSN is deployed.  For example, the working infrastructure of an oil rig will be composed by several floors, where the ceil and the walls could be covered with steel piping, as showed in the experiment of Luo et al.\cite{luo2011rf}.  On the other hand, a mining industry has a completly different setting, where the propagation characteristics of electromagnetic waves in underground mines are different from those in free space, because of the physical characteristics of the ore and the tunnel itself\cite{farjow2015novel,sun2010channel, grote2009wireless}.

This paper will present the attempts on modeling the wireless channels for WSN in industrial environments, with an emphasis on the mining industry.
Section I will present a brief explanation of a WSN, with emphasis on the protocol IEEE 802.15.4, the ZigBee Alliance and the characteristics of the physical layer. Section II will detail the characteristics that degradate the signal in a WSN link. In Section III the models that try to represent the industrial environment and the underground mining environment are presented, along with their results.  Finally, Section IV present the discusion and conclusions.

\section{Wireless Sensor Networks}
Wireless sensor networks are systems that comprise large populations (hundreds or thousands) of wirelessly connected heterogeneous sensor nodes that are physically small, inexpensive and spatially distributed across a large field of interest. Moreover, they consume little power to allow prolong operation for years\cite{low2005wireless}. 
\subsection{IEEE 802.15.4}

IEEE 802.15.4 is a technical standard which defines the operation of low-rate wireless personal area networks (LR-WPANs). It specifies the two first layers of the OSI model, i.e., physical layer and media access control.
\subsubsection{Physical Layer}
IEEE 802.15.4 defines the physical layer with three possible operational frequencies, each one with different bitrate, channels and modulation scheme, which are presented in table \ref{tab:ieee}.  Additionally, the standard uses direct sequence spread spectrum (DSSS) modulation.

\begin{table}[b]
\label{tab:ieee}
  \centering
	\begin{tabular}{ | c | c | c | c | }
	  \hline
	  Characteristic & Europe & USA & Worldwide \\
	  \hline
	  Frequency Assignment & 868 to 868.6 MHz & 902 to 928 MHz & 2.4 to 2.4835 GHz\\
	  \hline
	  Number of Channels & 1 & 10 & 16\\
	  \hline
	  Channel Bandwidth & 600 kHz & 2 MHz & 5 MHz\\
	  \hline
	  Data Rate & 20 kbps & 20 kbps & 250 kbps\\
	  \hline
	  Modulation & BPSK & BPSK & O-QPSK\\
	  \hline
	\end{tabular}
	\caption{Definitions of the PHY layer for IEEE 802.15.4}
\end{table}

\subsubsection{MAC Layer}
With regard to channel access, 802.15.4 uses carrier sense multiple access with collision avoidance (CSMA-CA). This multiplexing approach lets multiple users or nodes access the same channel at different times without interference. Most transmissions are short packets that occur infrequently for a very low duty cycle ($\leq 1$ \%), minimizing power consumption. 
\subsection{ZigBee Alliance}
The most widely deployed enhancement to the 802.15.4 standard is ZigBee, which is a standard of the ZigBee Alliance. The organization maintains, supports, and develops more sophisticated protocols for advanced applications. It uses layers 3 and 4 to define additional communications features. These enhancements include authentication with valid nodes, encryption for security, and a data routing and forwarding capability that enables mesh networking. The most popular use of ZigBee is wireless sensor networks using the mesh topology\footnote{http://www.zigbee.org/zigbee-for-developers/zigbee/}.

\section{Noise, fading and interference in the Industrial Wireless Channel}
\subsection{Pathloss and fading}
Cheffena defined very well the main sources of interference, noise and fading for an Industrial Wireless Channel (IWC).  In particular, in his study\cite{cheffena2016propagation, cheffena2012industrial}, he propose that one of the characteristics for an IWC is the heavy multipath propagation.  Additionaly, he states that no clear relationship between path-loss exponent and frequency can be established for many industrial environments\cite{cheffena2016propagation}.
\subsection{Noise}
Usually, the noise in a wireless communication systems is characterized as AWG.  However, in harsh factory environments, wireless systems are also affected by impulsive noise\cite{low2005wireless,cheffena2016propagation, blackard1993measurements}.
The figure \ref{fig:impulsive}

\begin{figure}[h] 
\centering
\includegraphics[scale=0.4]{noise.png} 
\caption{Resultado de las simulaciones para el modelo de la página web.}
\label{fig:web}
\end{figure}

\section{Models for Industrial Wireless Channel}
\subsection{Industrial Environment}
\subsection{Minning Environment}

\section{Summary and Conclusions}



\bibliographystyle{unsrt}
\bibliography{bibliografia}

\end{document}
